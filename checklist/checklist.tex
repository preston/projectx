\documentclass[10pt]{article}
%\usepackage{geometry}                % See geometry.pdf to learn the layout options. There are lots.
%\geometry{letterpaper}                   % ... or a4paper or a5paper or ... 
%\geometry{landscape}                % Activate for for rotated page geometry
%\usepackage[parfill]{parskip}    % Activate to begin paragraphs with an empty line rather than an indent
%\usepackage{graphicx}
%\usepackage{amssymb}
%\usepackage{epstopdf}
%\DeclareGraphicsRule{.tif}{png}{.png}{`convert #1 `dirname #1`/`basename #1 .tif`.png}

\usepackage{fancyhdr}
\usepackage{lastpage}

\pagestyle{fancy}
\fancyfoot[L]{Project X \\ CST 200}
\fancyfoot[C]{Presentation Checklist \\ {\copyright} Preston Lee 2010. All rights reserved.}
\fancyfoot[RO, LE] {{\thepage} of \pageref{LastPage}}

\title{Project X: Presentation Checklist}

\author{Preston Lee}
\date{}                                           % Activate to display a given date or no date

\begin{document}
\maketitle

\section{Abstract}
Congratulations! You're nearing your first software deliverable, and success is yours to lose. {\bf Complete and turn in this packet by labs end for an easy 4 points.}

\vspace{.5cm}

\hspace{.5in}{\bf Team Captain: }

\hspace{.5in}{\bf Team Code Name: }

\hspace{.5in}{\bf Project Code Name: }


\section{Presentation}
Use this checklist to make sure you're prepared!
\begin{description}
\item {\bf [  ] Running} Project is in a runnable state.
\item {\bf [  ] General Grading} You have met (or will have met) the custom grading criteria.
\item {\bf [  ] Specific Grading} You have met (or will have met) the custom grading criteria.
\item {\bf [  ] Presentation} You've prepared a 10-15 minute presentation and demo of your system, delivered by your captain. This should be {\bf prepared and professional}. Slides are optional, though a visual aid other than your application is highly recommended.
\item {\bf [  ] Equipment} Test connecting your machine(s) to the projector system. Both VGA and DVI projector connections will be provided.
\item {\bf [  ] Submission} {\it All} of your project materials must be submitted via a Dropbox folder. (See Blackboard for instructions.) I will pull your materials for final grading no later than Friday.
\item {\bf [  ] Practice} While your captain will be leading your presentation, I {\it may} ask questions to other members. Practice the fluidity and timing of your presentation and make sure {\it everyone} is prepared.
\end{description}




\pagebreak


\subsection{Document team contributions.}
Fill in the following table with the contributions of each team member.
\begin{table}[htdp]
\caption{Team member information.}
\begin{center}
\begin{tabular}{c c c c}
\hline\hline
Number & Name & Previous Course & Contributions \\ [.5ex]
\hline
(e.g.) & Alice Anderson  & CST 100 (Java) & (Several sentences, please.) \\
\hline
{\bf \#1} & & &  \\ [10ex]
{\bf \#2} & & & \\ [10ex]
{\bf \#3} & & & \\ [10ex]
{\bf \#4} & & & \\ [10ex]
{\bf \#5} & & & \\ [10ex]
\hline
\end{tabular}
\end{center}
\label{table:team}
\end{table}

\pagebreak

\section{Retrospective}

What were the biggest {\bf technological challenges} your team faced? What will you do differently next time?

\vspace{5cm}

What were the biggest {\bf communication challenges} your team faced? What will you do differently next time?

\vspace{5cm}

What were the biggest {\bf timing challenges} your team faced? What will you do differently next time?

\vspace{5cm}

\pagebreak

Did team members all make the same levels of contributions? Why? Is this a good thing or a bad thing... or both?

\vspace{5cm}

Did the project develop as you initially expected? Why?

\vspace{5cm}

Would a Waterfall or Iterative development methodology have worked better on your project? Why?

\vspace{5cm}



\pagebreak
\section{Final Project Grade}

(Summary of general requirements provided for convenience.)

\begin{itemize}
\item[Timeliness] Real-world projects can't wait until the last minute, and neither should you. The development of your project should be paced throughout the time you are given, {\bf not} crammed into the last week.
\item[Abstraction] Your must ``reuse" code (in a meaningful way) by using inheritance and well as polymorphic references where sensical in your application.
\item[Input] The application must accept input from the user, disk, network or other external source.
\item[Output] Programs must provide meaningful real-time output to the use. It's ok to write to disk, but you still must print to the screen or put up some sort of GUI. ({\bf Bonus}: Implement a GUI using Swing or SWT.)
\item[Exception Handling] Any/All user, disk, network etc. I/O must be ``solid": written with thorough exception handling practices to prevent provide a reasonable level of application robustness. Use try/catch and ({\bf Bonus}: Make use of a network connection.)
\item[Code Documentation] Code with documentation on usage and programmer thinking is a magnitude more valuable than code without. All code must be thoroughly comment in JavaDoc format. ({\bf Bonus}: Provide ``howto" documentation on all reusable classes.)
\item[Automated Test Cases] Provide a suite of "unit tests" that allow you to quickly run regression tests against your code base. You must also include ``negative'' test cases: code which intentionally calls functions with invalid input to verify that the code fails the way it is supposed to. For example, passing null, invalid numbers etc. to functions expecting ``correct" input should do something reasonable. ({\bf Bonus}: Use a unit test framework.) 
\end{itemize}

In the space below, please restate your approved custom grading criteria.
\begin{itemize}
\item[{\bf \#1}]
\item[{\bf \#2}]
\item[{\bf \#3}]
\item[{\bf \#4}]
\item[{\bf \#5}]
\item[{\bf \#6}]
\end{itemize}

\pagebreak
\section{Instructor Feedback}

(Do not write in this section.)

\end{document} 
